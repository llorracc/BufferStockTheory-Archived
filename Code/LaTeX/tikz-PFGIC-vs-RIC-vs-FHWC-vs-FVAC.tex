\documentclass{econtex}
%\renewcommand{\familydefault}{\sfdefault}
\usepackage{econtexShortcuts}\usepackage{econtexSetup}
% Define the handles for the various conditions
\newcommand{\PFGIC}{{\mbox{PF-GIC}}}
\newcommand{\AIC}{{\mbox{AIC}}}
\newcommand{\PFFVAC}{{\mbox{PF-FVAC}}}
\newcommand{\FVAC}{{\mbox{FVAC}}}
\newcommand{\GIC}{{\mbox{GIC}}}
\newcommand{\RIC}{{\mbox{RIC}}}
\newcommand{\WRIC}{{\mbox{WRIC}}}
\newcommand{\FHWC}{{\mbox{FHWC}}}

\usepackage{tikz, environ, etoolbox}
\usetikzlibrary{cd, external}
\tikzexternalize
% activate the following such that you can check the macro expansion in
% *-figure0.md5 manually
%\tikzset{external/up to date check=diff}

\def\temp{&} \catcode`&=\active \let&=\temp

\newcommand{\mytikzcdcontext}[2]{
  \begin{tikzpicture}[baseline=(maintikzcdnode.base)]
    \node (maintikzcdnode) [inner sep=0, outer sep=0] {\begin{tikzcd}[#2]
        #1
    \end{tikzcd}};
  \end{tikzpicture}}

\NewEnviron{mytikzcd}[1][]{%
% In the following, we need \BODY to expanded before \mytikzcdcontext
% such that the md5 function gets the tikzcd content with \BODY expanded.
% Howerver, expand it only once, because the \tikz-macros aren't
% defined at this point yet. The same thing holds for the arguments to
% the tikzcd-environment.
\def\myargs{#1}%
\edef\mydiagram{\noexpand\mytikzcdcontext{\expandonce\BODY}{\expandonce\myargs}}%
\mydiagram%
}

\begin{document}

\begin{equation}
\begin{mytikzcd}[row sep=huge, column sep=huge]
   \textbf{\Thorn} \ar{r}{\mathrm{\PFGIC}} \ar{d}{\mathrm{\RIC}} \ar{dr}{\mathrm{\PFFVAC}}& \Gamma \ar{d}{\mathrm{\FHWC}} \\
   \mathsf{\Rfree} &  \mathsf{\Rfree}^{1/\CRRA}\Gamma^{1 - 1/\CRRA} \ar{l}{\mathrm{\FHWC}}
 \end{mytikzcd}
\end{equation}

\end{document}


% [row sep=large, column sep=large]
%  \textbf{\Thorn} \ar{rr}{\mathrm{\PFGIC}} \ar{dr}{\mathrm{\RIC}}& & \Gamma \ar{dl}{\mathrm{\FHWC}} \\
%  & \mathsf{\Rfree} &

%  \textbf{\Thorn} \ar{r}{\mathrm{\PFGIC}} \ar{d}{\mathrm{\RIC}} \ar{dr}{\mathrm{\PFFVAC}}& \Gamma \ar{d}{\mathrm{\FHWC}} \\
%   \mathsf{\Rfree} &  \mathsf{\Rfree}^{1/\CRRA}\Gamma^{1 - 1/\CRRA} \ar{l}{\mathrm{\FHWC}}
