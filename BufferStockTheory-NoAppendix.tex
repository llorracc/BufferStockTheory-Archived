\documentclass[BufferStockTheory]{subfiles}\providecommand{\econtexRoot}{}
\renewcommand{\econtexRoot}{..}

% WARNING: Different execution depending on whether
% 0. Being compiled as standalone document
% * Compile this file, then main, then this one again
% * Keep iterating until neither file changes
% 0. Being compiled as subfile of main document
% * Not clear what proper order is; might NOT matter

\onlyinsubfile{\externaldocument{BufferStockTheory}} % Get xrefs -- esp to appendix -- from main file; only works properly if main file has already been compiled; 

\begin{document}
\providecommand{\versn}{} % Version; like, web or pdf or journal submission
\ifthenelse{\boolean{ifWeb}}{  \renewcommand{\ushort}{\underline}\renewcommand{\versn}{Web} }{} % ushort does not work in tex4ht

\hfill{\tiny \jobname~\versn~\today, time:\DTMcurrenttime, source-commit: \input{.git-source-commit}, public-commit: \input{.git-public-commit}}

%%%\begin{verbatimwrite}{BufferStockTheory.title}  % Write title to .title file
%%%Theoretical Foundations of Buffer Stock Saving
%%%\end{verbatimwrite}

\title{Theoretical Foundations of \\ Buffer Stock Saving}

\author{Christopher D. Carroll\authNum}

\keywords{Precautionary saving, buffer stock saving, marginal propensity to consume, permanent income hypothesis}

\jelclass{D81, D91, E21}

\renewcommand{\forcedate}{September 28, 2020}
\date{\forcedate}

\maketitle %% Version finally sent back to QE, as MS 354
\hypertarget{abstract}{}
\begin{abstract}
This paper builds theoretical foundations for rigorous and intuitive understanding of `buffer stock' saving models, pairing each theoretical result with a quantitative exploration.  After describing conditions under which the consumption function converges, the paper shows that a `target' buffer stock exists only under conditions strictly stronger than those that guarantee convergence of the consumption and value functions.  It also shows that the average growth rate of consumption equals the average growth rate of permanent income (in a small open economy populated by buffer stock savers).  Together, the (provided) numerical tools and (proven) analytical results constitute a comprehensive toolkit for understanding buffer stock models.
\end{abstract}

% Various resources 
\hypertarget{links}{}
\begin{small}
\parbox{\textwidth}{
\begin{center}
\begin{tabbing}
\texttt{~Archive:~} \= \= \url{http://www.econ2.jhu.edu/people/ccarroll/BufferStockTheory.zip} \\  %
\texttt{~~~~~PDF:~} \> \> \url{http://llorracc.github.io/BufferStockTheory/BufferStockTheory.pdf} \\
\texttt{~~Slides:~} \> \> \url{http://llorracc.github.io/BufferStockTheory/BufferStockTheory-Slides.pdf} \\
\texttt{~~~~~Web:~} \> \> \url{http://llorracc.github.io/BufferStockTheory/BufferStockTheory/}    \\
\texttt{Appendix:~} \> \> \url{http://llorracc.github.io/BufferStockTheory/BufferStockTheory\#Appendices}    \\
%\texttt{~~bibtex:~} \> \> \url{http://llorracc.github.io/BufferStockTheory/BufferStockTheory.bib}  \\
\texttt{~~GitHub:~} \> \> \url{http://github.com/llorracc/BufferStockTheory} \\
\texttt{~~~~~~~~~~} \> \> \textit{(In GitHub repo, see \texttt{/Code} for tools for solving and simulating the model)} \\
\end{tabbing}
\end{center}
          
\href{https://econ-ark.org/materials/BufferStockTheory}{CLICK HERE} for an interactive \href{https://en.wikipedia.org/wiki/Project\_Jupyter\#Jupyter_Notebook}{Jupyter Notebook} that uses the \href{https://econ-ark/HARK}{Econ-ARK/HARK} toolkit to produce all of the paper's figures (warning: the notebook may take several minutes to launch).  Information about citing the toolkit can be found at \href{https://econ-ark.org/acknowledging/}{Acknowleding Econ-ARK}.
} % end parbox{\textwidth}
\end{small}

\begin{authorsinfo}
\name{Contact: \href{mailto:ccarroll@jhu.edu}{\texttt{ccarroll@jhu.edu}}, Department of Economics, 590 Wyman Hall, Johns Hopkins University, Baltimore, MD 21218, \url{http://econ.jhu.edu/people/ccarroll}, and National Bureau of Economic Research.}
\end{authorsinfo}

\thanks{All numerical results herein were produced using the \href{https://econ-ark/HARK}{Econ-ARK/HARK} toolkit, which can be cited per our references (\cite{carroll_et_al-proc-scipy-2018}); for reference to the toolkit itself see \href{https://econ-ark.org/acknowledging/}{Acknowleding Econ-ARK}.  Thanks to James Feigenbaum, Joseph Kaboski, Miles Kimball, Qingyin Ma, Misuzu Otsuka, Damiano Sandri, John Stachurski, Adam Szeidl, Metin Uyanik, Weifeng Wu, Xudong Zheng,
  and Jiaxiong Yao for comments on earlier versions of this paper, John Boyd for help
  in applying his weighted contraction mapping theorem, Ryoji
  Hiraguchi for extraordinary mathematical insight that improved the
  paper greatly, David Zervos for early guidance to the literature,
  and participants in a seminar at Johns Hopkins University and a
  presentation at the 2009 meetings of the Society of Economic
  Dynamics for their insights.}

\titlepagefinish


\newtheorem{defn}{Definition}
\newtheorem{theorem}{Theorem}

\hypertarget{Introduction}{}
\section{Introduction}

\label{sec:intro}

%%%Following the success of Modigliani and Brumberg's~\citeyearpar{modigliani&brumberg:lifecycle} Life Cycle model and Friedman's~\citeyearpar{friedmanATheory} Permanent Income Hypothesis, a vast literature in the 1960s and 1970s formalized the idea that household spending can be modeled as reflecting optimal intertemporal choice.  Bewley~\citeyearpar{bewleyPIH} and \cite{seIncFluct} capped this literature, paving the way for the ascendancy of dynamic stochastic optimizing models in economics.



In the presence of empirically realistic transitory and permanent shocks to income \textit{a la} \cite{friedmanATheory}, only one other ingredient is required to define a testable model of optimal consumption: A description of preferences.  Modelers usually assume geometric discounting of a constant relative risk aversion (CRRA) utility function, because, starting with Zeldes~\citeyearpar{zeldesStochastic}, a large literature has shown that models of this kind have quantitative predictions that can match microeconomic evidence reasonably well.

A companion theoretical literature has shown that standard numerical solution methods provide good approximations to limiting ``true'' mathematical solutions -- but only for models more complex than the simple case with just shocks and utility.  The extra complexity has been required because standard contraction mapping theorems (beginning with \cite{bellmanDynamicProgramming} and including those following Stokey et.~al.~\citeyearpar{slpMethods}) cannot be applied when the utility function is unbounded (like CRRA - see \hyperlink{DiffFromLit}{section \ref{subsec:Setup}}).\footnote{It is unclear whether newer methods such as those of \cite{mnUnique} could overcome this problem, or how difficult it would be to do so; but in any case this particular problem does not seem to have been tackled by those methods or any others.}

This paper's first technical contribution is to articulate the (surprisingly loose) conditions under which the simple problem (without convenient shortcuts like a consumption floor or liquidity constraints) defines a contraction mapping with a nondegenerate consumption function (the main requirement is a \hyperlink{FVAC}{`Finite Value of Autarky'} condition).  Another contribution is to specify the conditions under which the resulting consumption function implies there is a `target' wealth-to-permanent-income ratio (so the model exhibits `buffer stock' saving behavior.)  The key requirement for existence of a target is that the model's parameters satisfy a \hyperlink{GIC}{``Growth Impatience Condition''} (equation \eqref{eq:GIC}) that relates preferences and uncertainty to the predictable growth rate of income.

\hypertarget{KMP}{}

Even without a formal proof, target saving of this kind has been intuitively understood to underlie central numerical results from the heterogeneous agent macroeconomics literature; for example, the logic of target saving is central to the explanation by \cite{kmpHandbook} of the fact that, during the Great Recession, middle-class consumers cut their consumption more than the poor or the rich.  The theoretical logic articulated below explains this finding:  Learning that the future has become more uncertain does not change the urgent imperatives of the poor (their high $\uFunc^{\prime}(\cRat)$) because they have little room to maneuver.  Increased labor income uncertainty does not change the behavior of the rich because the increase in uncertainty does not threaten their consumption much.  Only people in the middle have both the motivation and the wiggle-room to reduce their discretionary spending.

Conveniently, elements required for the convergence proof turn out to provide analytical foundations for many other results that have become familiar from the numerical literature.  All theoretical conclusions are paired with numerically computed illustrations (using an open-source toolkit available from the \href{https://github.com/econ-ark/REMARK/blob/master/REMARKs/BufferStockTheory/BufferStockTheory.ipynb}{Econ-ARK} project).  All of the insights of this paper are instantiated in the toolkit, which algorithmically flags parametric choices under which a problem fails to define a contraction mapping, under which a target level of wealth does not exist, or under which the solution is otherwise degenerate.

Thus, the theoretical foundations provided here are valuable both because they provide intuition about the determinants of saving targets, and because they make it easier to develop reliable numerical solution methods (by providing tight restrictions that valid solutions must satisfy).

The paper proceeds in three parts.

The first part articulates the \hyperlink{Sufficient-Conditions}{conditions required} for the problem to define a unique nondegenerate limiting consumption function, and discusses the relation of the paper's model to models previously considered in the literature.  The required conditions are interestingly parallel to those required for the \hyperlink{Factors-Defined-And-Compared}{liquidity constrained perfect foresight model}; that parallel is explored and explained.  Next, the paper derives some limiting properties of the consumption function as cash approaches infinity and as it approaches its lower bound, and the theorem is proven explaining when the problem defines a contraction mapping.  Finally, a related class of commonly-used models (exemplified by Deaton~\citeyearpar{deatonLiqConstr}) is shown to constitute a particular limit of this paper's more general model.

The \hyperlink{AnalysisoftheConvergedConsumptionFunction}{next section} examines five key properties of the model. First, as \hyperlink{LimitsAsmtToInfty}{cash approaches infinity} the expected growth rate of consumption and the marginal propensity to consume (MPC) converge to their values in the perfect foresight case. Second, as \hyperlink{LimitsAsmtToZero}{cash approaches zero} the expected growth rate of consumption approaches infinity, and the MPC approaches a simple analytical limit.  Third, if the consumer is `growth impatient,' a \hyperlink{onetarget}{unique target cash-to-permanent-income ratio} will exist.  Fourth, at the target cash ratio, the \hyperlink{cGroLTpGro}{expected growth rate of consumption} is slightly less than the expected growth rate of permanent noncapital income.  Finally, the expected growth rate of consumption \hyperlink{dcgdxneg}{is declining in the level of cash}. The first four propositions are proven under general assumptions about parameter values; the last is shown to hold if there are no transitory shocks, but may fail in extreme cases if there are both transitory and permanent shocks.

Szeidl~\citeyearpar{szeidlInvariant} has shown that such an economy will be characterized by stable invariant distributions for the consumption ratio, the wealth ratio, and other variables.\footnote{Szeidl's proof supplants the analysis in an earlier draft of this paper, which conjectured that such a result held and provided supportive simulation evidence.}  Using Szeidl's result, the final section discusses conditions under which, even with a fixed aggregate interest rate that differs from the time preference rate, an economy populated by buffer stock consumers converges to a balanced growth equilibrium in which the growth rate of consumption tends toward the (exogenous) growth rate of permanent income.

\hypertarget{The-Problem}{}
\section{The Problem}

\subsection{Setup}
\label{subsec:Setup}

The consumer solves an optimization problem from period
$t$ until the end of life at $T$ defined by the objective
\begin{verbatimwrite}{\EqDir/supfn.tex}
\begin{align}
  \label{eq:supfn}
  \max~ \Ex_{t}\left[\sum_{n=0}^{T-t} \DiscFac^{n} \uFunc(\cLevBF_{t+n})\right]
\end{align}
\end{verbatimwrite}
  \begin{align*}%    \label{eq:supfn}
    \max~ \Ex_{t}\left[\sum_{n=0}^{T-t} \DiscFac^{n} \uFunc(\cLevBF_{t+n})\right]
  \end{align*}

where
\begin{align}
  \uFunc(\bullet)=\bullet^{1-\CRRA}/(1-\CRRA) \label{eq:crrautil}
\end{align}
 is a constant relative risk aversion utility function with $\CRRA > 1$.\footnote{The main
  results also hold for logarithmic utility which is the limit as
  $\CRRA \rightarrow 1$ but incorporating the logarithmic special case
  in the proofs is cumbersome and therefore
  omitted.}$^{,}$\footnote{We will define the infinite horizon
  solution as the limit of the finite horizon problem as the horizon
  $T-t$ approaches infinity.}  The consumer's initial condition is
defined by market resources $\mLevBF_{t}$ (which \cite{deatonLiqConstr}
called `cash-on-hand') and permanent noncapital income $\pLevBF_{t}$.

In the usual treatment, a dynamic budget constraint (DBC) simultaneously incorporates
all of the elements that determine next period's $\mLevBF$ given this
period's choices; but for the detailed analysis here, it will be useful to
disarticulate the steps so that individual ingredients can be separately examined:

\begin{verbatimwrite}{\EqDir/DBCparts.tex}
\begin{align}
\aLevBF_{t}    & = \mLevBF_{t}-\cLevBF_{t}  \label{eq:DBCparts} \\
\bLevBF_{t+1}    & = \aLevBF_{t} \Rfree \notag \\
\pLevBF_{t+1}  & = \pLevBF_{t} \underbrace{\PGro\pShk_{t+1}}_{\equiv \PGro_{t+1}}  \notag \\
\mLevBF_{t+1}  & =  \bLevBF_{t+1} +\pLevBF_{t+1}\tShkAll_{t+1},  \notag
\end{align}
\end{verbatimwrite}

\end{document}


