\providecommand{\econtexRoot}{}
\renewcommand{\econtexRoot}{..}
\documentclass[\econtexRoot/BufferStockTheory]{subfiles}
\providecommand{\econtexRoot}{}
\renewcommand{\econtexRoot}{..}
\onlyinsubfile{% https://tex.stackexchange.com/questions/463699/proper-reference-numbers-with-subfiles
    \csname @ifpackageloaded\endcsname{xr-hyper}{%
      \externaldocument{\econtexRoot/BufferStockTheory}% xr-hyper in use; optional argument for url of main.pdf for hyperlinks
    }{%
      \externaldocument{main}% xr in use
    }%
    \renewcommand\labelprefix{}%
    % Initialize the counters via the labels belonging to the main document:
    \setcounter{equation}{\numexpr\getrefnumber{\labelprefix eq:AAgg}\relax}% eq:AAgg is the last numbered equation in the main text; start counting up from there
}
\begin{document}
\subsection{Convergence of $\vFunc_{t}$ in Euclidian Space}\label{sec:vEuclidian}

Boyd's theorem shows that $\TMap$ defines a contraction mapping
in a $\phiFunc$-bounded space. We now show that $\TMap$ also
defines a contraction mapping in Euclidian space.

Calling $\vFunc^{\ast}$ the unique fixed point of the operator $\TMap$, since $\vFunc^{\ast }(\mRat)=\TMap {\vFunc}^{\ast }(\mRat)$,
\begin{equation}
\left\Vert \vFunc_{T-n+1}-\vFunc^{\ast }\right\Vert _{\phiFunc }\leq \Shrinker
^{n-1}\left\Vert \vFunc_{T}-\vFunc^{\ast }\right\Vert _{\phiFunc }.
\end{equation}%
On the other hand, $\vFunc_{T}-\vFunc^{\ast }\in \mathcal{C}_{\phiFunc }\left( \mathscr{A},
\mathscr{B}\right) $ and $\MPC =\left\Vert \vFunc_{T}-\vFunc^{\ast }\right\Vert_{\phiFunc }<\infty $ because $\vFunc_{T}$ and $\vFunc^{\ast }$ are in $\mathcal{C}_{\phiFunc
}\left( \mathscr{A},\mathscr{B}\right) $. It follows that%
\begin{equation}
\left\vert \vFunc_{T-n+1}(\mRat)-\vFunc^{\ast }(\mRat)\right\vert \leq \MPC \Shrinker^{n-1}\left\vert \phiFunc(\mRat)\right\vert.
\end{equation}%
Then we obtain
\begin{equation}
\underset{n\rightarrow \infty }{\lim }\vFunc_{T-n+1}(\mRat)=\vFunc^{\ast }(\mRat).
\end{equation}

Since $\vFunc_{T}(\mRat)=\frac{\mRat^{1-\CRRA }}{1-\CRRA }$, $\vFunc_{T-1}(\mRat)\leq \frac{\left(\MaxMPC \mRat\right)^{1-\CRRA}}{1-\CRRA}<\vFunc_{T}(\mRat)$. On the other hand, $\vFunc_{T-1}\leq \vFunc_{T}$
means $\TMap\vFunc_{T-1}\leq \TMap\vFunc_{T}$, in other words, $\vFunc_{T-2}(\mRat)\leq \vFunc_{T-1}(\mRat)$.
Inductively one gets $\vFunc_{T-n}(\mRat)\geq \vFunc_{T-n-1}(\mRat)$. This means that $\left\{
\vFunc_{T-n+1}(\mRat)\right\} _{n=1}^{\infty }$ is a decreasing sequence,
bounded below by $\vFunc^{\ast}$.


\subsection{Convergence of $\cFunc_{t}$}
\label{subsec:cConverges}%\label{sec:cEuclidian}

Given the proof that the value functions converge, we now show the
pointwise convergence of consumption functions
$\left\{\cFunc_{T-n+1}(\mRat)\right\} _{n=1}^{\infty }$.

Consider any convergent subsequence $\left\{\cFunc_{T-n(i)}(\mRat)\right\}$ of $\left\{\cFunc_{T-n+1}(\mRat)\right\} _{n=1}^{\infty }$ converging to $c^*$. By the definition of $\cFunc_{T-n}(\mRat)$, we have 
\begin{equation}
    \uFunc(c_{T - n(i)}(\mRat)) + \DiscFac \Ex_{T - n(i)}[{\PGro}_{T-n(i)+1}^{1 - \CRRA}\vFunc_{T-n(i)+1}(\mRat)] \geq \uFunc(c_{T - n(i)}) + \DiscFac \Ex_{T - n(i)}[{\PGro}_{T-n(i)+1}^{1 - \CRRA}\vFunc_{T-n(i)+1}(\mRat)],
\end{equation}
for any $\cRat_{T - n(i)} \in [\MinMinMPC \mRat,\MaxMPC \mRat]$. Now letting $n(i)$ go to infinity, it follows that the left hand side converges to $\uFunc(\cRat^*) + \DiscFac \Ex_{t}[{\PGro}_{t}^{1 - \CRRA}\vFunc(\mRat)]$, and the right hand side converges to $\uFunc(\cRat_{T - n(i)}) + \DiscFac \Ex_{t}[{\PGro}_{t}^{1 - \CRRA}\vFunc(\mRat)]$. So the limit of the preceding inequality as $n(i)$ approaches infinity implies 
\begin{equation}
   \uFunc(\cRat^*) + \DiscFac \Ex_{t}[{\PGro}_{t + 1}^{1 - \CRRA}\vFunc(\mRat)] \geq \uFunc(\cRat_{T - n(i)}) + \DiscFac \Ex_{t}[{\PGro}_{t + 1}^{1 - \CRRA}\vFunc(\mRat)].
\end{equation}
Hence, $c^* \in \underset{\cRat_{T - n(i)} \in [\MinMinMPC \mRat,\MaxMPC \mRat]}{\arg \max }\left\{ \uFunc(\cRat_{T - n(i)}) + \DiscFac \Ex_{t}[{\PGro}_{t + 1}^{1 - \CRRA}\vFunc(\mRat)]\right\}$. By the uniqueness of $\cFunc(\mRat)$, $c^* = \cFunc(\mRat)$.



\end{document}


